\chapter{Grundlagen}

  \section{Android SDK}
  
  Android ist das Linux-basierte Betriebssystem f�r mobile Endger�te, welches
  von Google 2011 offiziell zur Verf�gung gestellt wurde. Android selbst gilt
  als sogenannte frei Software, welche bis auf den System-Kern unter der
  Apache-Lizens steht. Diese Tatsache unter Anderen erm�glichte eine rasante
  Verbreitung dieses Betriebsystems auf vielen Ger�ten unterschiedlicher
  Hersteller.
  
  \begin{wrapfigure}{l}{0.3\textwidth}
   \centering
   \includegraphics[width=0.25\textwidth]{chapter2/android}
   \caption[Von Text umflossenes Bild]{Android}
   \label{fig:Wrap}
  \end{wrapfigure}

 Somit waren im M�rz 2013 etwa 750 Millionen Android End-Ger�te aktiviert und
 man merkt schnell, dass die Popularit�t dieses Betriebssystem immer mehr
 zunimmt.Da ein Smartphone Betriebsystem auch von den angeboten Apps lebt hat
 Google um die Entwicklung dieser eine Entwicklungswerkzeugsammlung zur
 Verf�gung gestellt, welche die Entwicklung von Applikationen f�r Android
 m�glichst einfach erm�gichen soll. Bei dieser Werkzeugensammlung handelt es
 sich um das Android SDK, welches auch als das Android Developer Tool kurz ADT
 verf�gbar ist. ADT ist ein Plugin f�r das mittlerweile sehr weit verbreitete
 Entwicklungsumgebung Eclipse, welche die Entwicklung und die �bertragung der
 Applikation auf das Ger�t m�helos erm�glicht. Weiterhin bringt das Android SDK
 einen Emulator mit sich, welches das Testen von Apps unter unteschiedlichen
 Konfigurationen erm�glicht ohne dass man ein Android-Ger�t ben�tigt. Vorallem
 erm�glicht das Android SDK die Entwicklung der Apps in der Programmiersprache
 Java, welche sich immer h�herer Beliebtheit erfreut und dank Eclipse zu einer
 h�heren Produktivit�t beitr�gt.
 Als n�chstes gehe ich auf die einzelne Grundlagen von Android SDK ein um die
 Funktionalit�t dieser zu beschreiben.
 
 \subsection{Activity}
Eine Activity ist eine Klasse, welche einzelne UI-Fenster representiert. Somit
bestehen die meisten Apps in Android aus mehreren Activities, welche
mit einander verbunden sind. Ein Use-Case bei dem man mehrere Fenster hat, w�rde
sozusagen mehrere Activities nacheinander aufrufen. 
Um eine Activity zu erstellen muss seine Klasse eine Unterklasse von Activity
sein. Zus�tzlich muss man die Methode \textbf{\textit{onCreate()}} in seiner
Klasse �berschreiben. Diese Methode wird jedesmal aufgerufen wenn die Activity
erstellt wird, also auf dem Bildschirm erscheinen soll. In diese Methode kommen
meist Anpassungen an die UI hinein oder andere Operationen die beim Start
notwendig sind.

\subsection{Service}
Ein Service ist eine Komponente, welche dazu gedacht ist Hintergrundprozesse zu
�bernehmen. Ein Service wird von z.B. einer Activity gestarted und l�uft dabei
im Hintergrund, selbst wenn die Activity nicht mehr existiert. Somit bietet sich
ein Service gut an um z.B. die Netzwerkkommunikation im Hintergrund zu behandeln
ohne die Applikation selbst zu behindern.
Weiterhin l�sst sich ein Service auch an z.B eine Applikation binden, sodass der
Service auch beendet wird wenn die Applikation geschlossen wird.

\subsection{Application}
Application bietet zus�tzlich zu den Activities die M�glichkeit w�hrend der
ganzen Laufzeit der Applikation eine feste Instanz zu haben, welche den Zustand
bestimmter Daten beinhaltet. So kann man es im Prinzip mit einem Singleton
vergleichen der den Status der Applikation h�lt. Um an die Instanz zu kommen
muss man aus dem Kontext heraus die Methode
\textbf{\textit{Context.getApplicationContext()}} aufrufen.

\subsection{Context}
Der Context beinhaltet Informationen �ber die Applikationsumgebung und l�sst
verschiedene Aktionen zu, wie z.B. das Aufrufen von weiteren Activities. Eine
Activity ist eine Unterklasse vom Context und wird bei der Erstellung von z.B.
einer View an diese �bergeben.

  \section{AllJoyn}

    Duis\cite{Lor09} sed lectus sem. Proin viverra venenatis tincidunt. Fusce eget turpis sit amet erat vestibulum pretium. In hac habitasse platea dictumst. Morbi eget massa et ante laoreet iaculis. Duis vitae nulla nulla. Suspendisse sit amet diam at enim accumsan consequat a eu sem. Proin venenatis ullamcorper gravida. Proin fermentum, metus vitae hendrerit bibendum, ligula dolor vestibulum eros, sed molestie lectus urna eget libero. Aenean ut sem nec metus tristique pretium.
