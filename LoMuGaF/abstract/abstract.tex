\chapter*{Zusammenfassung}

% Inhaltsverzeichnis und Kopfzeile
\addcontentsline{toc}{chapter}{Zusammenfassung}
\markboth{Zusammenfassung}{Zusammenfassung}

  Diese Arbeit umfasst die Implementierung einer Bibliothek f�r Mehrspielerspiele auf
  Peer-to-Peer-Basis \ref{ptp}.
  Peer-to-Peer bedeutet das f�r die Kommunikation kein Server verwendet wird. Dies wird jedoch
  noch genauer im Kapitel \ref{sec:ptp} erl�utert. F�r die Realisierung
  der Verbindung zwischen den Ger�ten wurde das Rahmenwerk AllJoyn \ref{alljoyn}
  verwendet, welcher im Kapitel \ref{sec:alljoyn} erl�utert wird.
  Weiterhin wurden zwei Spiele mithilfe dieser Bibliothek implementiert
  und dienen als praktische Beispiele.
  Die Bibliothek ist f�r Android \ref{Android} Ger�te bestimmt und wird daher
  auf Basis des Android SDK \ref{andSdk} realisiert.
  Die Bibliothek sollte eine Kommunikation zwischen mehreren Ger�ten soweit
  vereinfachen, dass bei der Implementierung weiterer Spiele nur um die
  Spielmechanik Gedanken gemacht werden muss. Selbst die
  Lobby-Funktionalit�t, wie das Erstellen und das Verbinden zum Spiel, wird
  von der Bibliothek �bernommen.
  Im weiteren Verlauf wird die implementierte Bibliothek als \ib{PTPLibrary}
  genannt.
  
  \vspace{0.5cm}

  \textbf{Schlagw�rter:} Multiplayer, Android, Lokal, Peer-to-Peer
