\chapter{Ping Testapplikation}
In diesem Kapitel wird auf eine Applikation eingegangen, die dazu verwendet wird
die \ib{Latenzzeit} bei der Kommunikation zwischen Smartphones zu messen. Es
wird ein Android Projekt erstellt, dass wie bei den beiden Bespielen in Kapitel
\ref{chapter:graph} und \ref{chapter:maumau} erkl�rt, die PTPLibrary als
Bibliothek referenziert. Mit der Latenzzeit ist die Zeit gemeint, die zwischen
dem Versenden eines Signals und dem Empfangen einer Antwort verstreicht. Die
Latenzzeit wird auch als der \ib{Ping} bezeichnet. Dazu werden allen Smartphones
eine Nachricht geschickt, die in der selben Session sind. Die Session wird wiederum durch die vorimplementierte Lobby erstellt.
Die Behandlung der Nachrichten, wird mithilfe eine Handlers realisiert, der die
eingehenden Nachrichten an den UIThread weiterleitet. Der Codeausschnitt
\ref{ping} zeigt die beiden Methoden, die f�r das Messen verwendet werden.
\begin{lstlisting}[caption=Ping Methoden,label=ping]
private void ping(MessageInfoHolder obj) {
	PTPHelper.getInstance().sendDataToAllPeers(PingData.PING_REPLY, obj.data);
}

private void pingReply(MessageInfoHolder obj) {
	if(obj.data[0] != null && obj.data[1]!= null && obj.data[1].equals(PTPHelper.getInstance().getUniqueID())){			
		long timeStampSent = Long.valueOf(obj.data[0]);
		long timeSpent = System.currentTimeMillis() - timeStampSent;
		getPingData().addToPingTable(obj.sentBy, timeSpent);
	}
}
\end{lstlisting}
Bei den Methoden wird das \ib{MessageInfoHolder} Objekt verwendet, das dazu
ben�tigt wird die Daten der Nachricht als ein Objekt zusammenzufassen. Die
Methode \ib{ping()} wird ausgef�hrt, falls eine Pinganfrage ankommt. Dabei wird
der ganze Methodeninhalt wieder zur�ck geschickt. Die Methode \ib{pingReply}
wird aufgerufen, falls eine Pinganfrage zugekommen ist. Dabei wird auch gepr�ft
ob, die Pinganfrage dem Empf�nger geh�rt oder nicht, da die Anfrage an alle
Sessionteilnehmer geschickt wird. Daraufhin wird die Zeit, die mit der Nachricht
verschickt wurde mit der aktuellen Zeit verglichen und daraus wird die Differenz
gebildet. Die Differenz entspricht der Latenzzeit. Diese wird in ein Datenmodel
gespeichert, das wiederrum die Ergebnisse als eine Liste auf dem Bildschirm
darstellen l�sst. Das Versenden der Pinganfragen wird in der eine
anderen Activity Klasse, durch das dr�cken eine Buttons realisiert. Der
Implementierung f�r das Versenden ist im Codeausschnitt \ref{sendPing} zusehen.
\begin{lstlisting}[caption=Sende Pinganfrage,label=sendPing]
long currentTime = System.currentTimeMillis();
PTPHelper.getInstance().sendDataToAllPeers(PingData.PING, new String[]{""+currentTime, PTPHelper.getInstance().getUniqueID()});
\end{lstlisting}
Dabei wird wie aktuelle Zeit und die ID des Versenders an alle Sessionteilnehmer
versendet. Die Ergebnisse der Messung sind auf den Bildern
\ref{fig:ping} a und b zu sehen.
\begin{figure}[hbt!]
\centering
	\begin{subfigure}[b]{0.4\textwidth}
	    \centering
	    \includegraphics[width=\textwidth]{chapter6/samsungping}
	    \caption{Samsung Galaxy S3}
    \end{subfigure}
	\begin{subfigure}[b]{0.4\textwidth}
	    \centering
	    \includegraphics[width=\textwidth]{chapter6/sonyping}
	    \caption{Sony Xperia Tipo ST21i}
    \end{subfigure}
    \caption{Ping Messergebnisse}
    \label{fig:ping}
\end{figure} 
Diese Bilder zeigen Bildschirmausschnitte der Ping Application auf den
jeweiligen Smartphone. Dabei ist zu erkennen, dass die Pingzahlen, die in
Millisekunden dargestellt sind, auf dem Samsung Smartphone ungleichm�ssig sind.
Dies ist darauf zur�ckzuf�hren, dass die Verarbeitung der Pinganfragen auf dem
Sony Smartphone stattfinden, das nur �ber einen Prozessorkern besitzt und
deswegen bei der Verarbeitung von mehreren Threads Verz�gerungen verursachen
kann. Das Samsung Galaxy S3 hingegen besitzt einen 4 Kern Prozessor und
verarbeitet die Pinganfragen recht schnell. Auf dem Bild \ref{fig:ping}
b sind die Antwortzeiten des Samsung Smartphones zu sehen, und diese zeigen
eine konstante Antwortzeit von etwa 40 Millisekunden.
