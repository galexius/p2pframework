\chapter{Einleitung}

Spiele auf mobilen Endger�ten gibt es unz�hlige. Jedoch sind die
meisten dieser Spiele entweder Solospiele, also Spiele f�r einen Spieler, oder
sind f�r mehrere Spieler, aber unter Verwendung eines Servers. Trotz der mittlerweile gut ausgebauten Netzabdeckung und guter Internetgeschwindigkeit auf Smartphones, kann das
Spielerlebniss durch eine schwankende Internetgeschwindigkeit getr�bt werden. So
sind z.B. Actionspiele meist nur �ber eine Wlan Verbindung spielbar. Au�erdem
gibt es Situationen bei denen man keine oder nur eine beschr�nkte Internetverbindung
hat, z.B. wenn man mit dem Zug unterwegs ist oder sich gerade in Ausland
befindet und sich die Roaminggeb�hren sparen will.
Um die Entwicklung solcher Spiele vorran zu treiben und es den Entwickler soweit
wie m�glich zu erleichtern, sollte im Rahmen dieser Arbeit eine Bibliothek
entwickelt werden. Es sollen die kommunikatonsbedingten Herausforderungen mit der
PTP-Library gel�st werden, sodass die Entwickler sich nur um die Implementierung
der Spiele selbst gedanken machen m�ssen. Die Entwickler h�tten nur die Aufgabe sich um die
Synchronisierung der spielbezogenen Daten zu k�mmern. Die �bertragen der
Daten an die einzelnen Ger�t w�rde von der PTP-Library �bernommen.
Damit die Entwicklung solcher Spiele f�r die Hobbyentwickler keine finanziellen
H�rden stellt, wurde die PTP-Library mit dem Android SDK \cite{andsdk}
entwickelt.
Da das Android SDK kostenlos f�r jeden Entwickler zur Verf�gung steht, kann jeder Entwickler, der
�ber ein Android Ger�t verf�gt gleich mit der Entwicklung des Spieles loslegen.
Nat�rlich l�sst sich die PTP-Library auch f�r praktische Anwendungen, wie
z.B. Textmessanger, verwenden, jedoch wird im Rahmen dieser Ausarbeitung nur
auf die Realisierung von Spielen eingegangen.

  
\section{L�sungsansatz}  
  
Um sich die ganzen Herausforderung bei der Realisierung von Peer-to-Peer
\cite{ptp} Kommunikation zu ersparen wurde ein weiteres Rahmenwerk namens
AllJoyn verwendet, welches die die Verbindung und die Kommunikation zum
gr��ten Teil schon �bernimmt. Somit blieb zum Einen eine geschickte
Integration von AllJoyn in die PTP-Library. Weiterhin mussten die vielen,
gerade auf den ersten Blick komplizierten, Konfigurationen des AllJoyn
Rahmenwerks durch eine einfache Schnittstelle erweitert werden. Die
Netzwerkverbindung l�uft in einem seperaten Thread und muss daher
mithilfe von \ib{Handlern} \ref{subsec:handler} von dem \ib{UI-Thread}
\ref{subsec:uithread} abgekoppelt werden, sodass sich die beiden Threads nicht
behindern. Als praktische Anwendungsbeispiele wurde ein Echtzeitpuzzlespiel und ein Zugbasiertes Kartenspiel namens MauMau implementiert.
   
\section{Gliederung}

Die Arbeit beinhaltet zu Anfang die Erkl�rung der verwendeten Rahmenwerke
AndroidSDK und AllJoyn. Dabei werden die Konzepte dieser Rahmenwerke grob
erl�utert. Dann wird das Konzept und die Implementierung der
PTP-Library, welches das Hauptaugenmerk dieser Arbeit ist,
beschrieben. Daraufhin wird beschrieben, wie die zwei Spielebeispiele realisiert
wurden, sowie Herausforderung, die w�hrend der Implementierung entstanden.


