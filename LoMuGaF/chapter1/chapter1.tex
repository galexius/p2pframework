\chapter{Einleitung}

Spiele f�r Smartphones gibt es unz�hlige. Die
meisten dieser Spiele sind entweder Solospiele,
oder sind f�r mehrere Spieler unter Verwendung eines Servers. Trotz der
mittlerweile gut ausgebauten Netzabdeckung und guter Internetgeschwindigkeit auf Smartphones, kann das Spielerlebniss durch eine schwankende Internetgeschwindigkeit getr�bt werden.
Dies ist von der Unterschiedlichen Netzabdeckung der einzelnen
Mobilfunkanbieter abh�ngig.\cite{coverage} Actionspiele, die eine schnelle
Antwortzeit ben�tigen, setzen somit mindestens eine \ib{HSDPA}\cite{hsdpa}
Verbindung vorraus.
Einige Spiele senden au�erdem eine gro�e Menge an Daten an den Server. Das kann,
abh�ngig vom vorhandenen Vertrag, das Datenvolumen schnell verbrauchen.
Es gibt auch Situtationen, bei denen es keine oder nur eine beschr�nkte Internetverbingung gibt, z.B. im Ausland oder bei einer
Zugfahrt. F�r diese Situtationen gibt es die M�glichkeit eine lokale
\ib{Peer-to-Peer}\cite{ptp} Verbindung zwischen den Smartphones zu erstellen.
Das Interesse eine lokale Verbingung zwischen Smartphones herstellen zu k�nnen,
wird auch an den Bem�hungen seitens Samsung erkennbar.
Samsung hat 2013, w�hrend der Erstellung dieser Arbeit, das \ib{Chord
SDK}\cite{chord} herausgebracht, das die Kommunikation zwischen einzelnen
Samsung Smartphones �ber das lokale Funknetz erm�glicht. Das Chord SDK steht
jedoch unter der Samsung License und erm�glicht nur die Entwicklung auf Samsung
Smartphones.
Die Entwicklung einer Bibliothek, die \ib{OpenSource} ist und die Verwendung auf
allen Android Ger�ten erlaubt, bietet somit eine gute Alternative. Durch diese
Bibliothek wird die Entwicklung der Spiele soweit vereinfacht, dass bei der
Entwicklung nur auf die Spiellogik geachtet werden muss. Die
Verbindungtechnischen Prozesse werden von der Bibliothek �bernommen.
F�r die Entwicklung eines Spiele auf Basis der Bibliothek wird nur ein
Android Ger�t ben�tigt.
Die Bibliothek kann auch f�r praktische Anwendungen, wie z.B. Textmessanger,
verwendet werden. Es wird jedoch im Rahmen dieser Ausarbeitung nur auf die
Realisierung von Spielen eingegangen.
Die Bibliothek wird im weiteren Verlauf der Arbeit als \ib{PTPLibrary}
bezeichnet.

  



