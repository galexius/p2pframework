\chapter{Fazit}
Die Entwicklung von Peer-to-Peer Spielen im Lokalen Wlan ist
eine herausfordende und spannende Sache. Mehrspieler Spiele ohne einen Server
sind heutzutage noch nicht sehr weit verbreitet und das bietet daher sehr viele
M�glichkeiten neue Konzepte und Ideen zu realisieren. Die Herausforderung liegt
in der Komplexit�t der Implementierung dieser Spiele, sowie in den
Beschr�nkungen, welche durch den gr��eren Synchronizationsaufwand entstehen.
Besonder an dem Praktischen Beispiel MauMau sah man, dass es viele F�lle gibt, die man beachten muss um den Zustand synchron zu halten und daher war es notwendig mehr Informationen �ber das Netzwerk zu verschicken, was zu mehr
Datenverkehr f�hrt. Das Rahmenwerk AllJoyn hielt was es versprach, mit all
seinen Funktionalit�ten, bis auf den leichten Einstieg, denn gerade das
Verstandniss des Mechanismus ist anfangs recht m�hsam und bedarf viel Ziel. Eine
weitere Problematik kommt aus der Tatsache, dass der Android Emulator, welcher
mit dem Android SDK mitgelieft wird, keine Wlan Unterst�tzung besitzt und somit
die Entwicklung mithilfe des Emulator nicht m�glich ist. Dies f�hrt dazu, dass
f�r die Entwicklung von Mehrspielerspielen mindestens 2 Android Ger�te
notwendig sind um das Spiel vern�nftig testen zu k�nnen. Daher war es auch
schwer Belastungstest durchzuf�hren, da daf�r eine entsprechende Anzahl an
Ger�ten ben�tigt wird. Alternativ g�be es noch die M�glichkeit eine Android x86
Version\cite{andx86} auf einer VM laufen zu lassen, jedoch ist die native
Bibliothek von AllJoyn nur f�r die ARM-Architektur verf�gbar. Doch auch hier
gibt es eine M�glichkeit AllJoyn f�r die x86 Architektur zu bauen, da der
Quellcode frei verf�gbar ist, was jedoch einen gro�en Aufwand bedeutet und die
Auseinandersetzung mit dem Android Quellcode. Trotz der Herausforderungen und
Probleme mit dem Peer-to-Peer Prinzip, besteht das Interesse f�r eine
unabh�ngige Kommunikation zwischen Mobilen Endger�ten.
Dies zeigt z.B. Samsung mit Samsung Chrod SDK\cite{chord}, welches im Prinzip
dem AllJoyn SDK �hnlich ist, bis auf die Tatsache, dass es
nicht Opensource ist und unter Samsungs License steht, was die Entwicklung etwas
einschr�nken kann. Aber das Interesse von Samsung best�tigt, dass ein gewisser
Potential in der Entwicklung von Peer-to-Peer Anwendung und Spielen besteht.
