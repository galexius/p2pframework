\chapter{Fazit}
Die Entwicklung von Peer-to-Peer Spielen im Lokalen Wlan ist
eine herausfordende und spannende Sache. Mehrspieler Spiele ohne einen Server
sind heutzutage noch nicht sehr weit verbreitet und das bietet daher sehr viele
M�glichkeiten neue Konzepte und Ideen zu realisieren. Die Herausforderung liegt
jedoch in der Komplexit�t der Implementierung dieser Spiele, sowie in den
Beschr�nkungen, die durch den gr��eren Synchronizationsaufwand entstehen.
Besonder an dem Praktischen Beispiel MauMau wird deutlich, dass es viele F�lle
gibt, die zu beachten sind, um den Zustand synchron zu halten. Daher war es
notwendig mehr Informationen �ber das Netzwerk zu verschicken, das wiederum zu
mehr Datenverkehr f�hrt. Das Kartenspiel ist jedoch ein Rundenbasiertes
Spiel, sodass eine gr��ere Datenmenge das Spielerlebniss nicht
beeintr�chtigt. Das Graphenspiel zeigt durch seine einfache Implementierung, wie
schnell ein Spiel auf Peer-to-Peer Basis realisiert werden kann. 
Das Rahmenwerk AllJoyn hielt was es versprach, mit all seinen
Funktionalit�ten, bis auf den leichten Einstieg. Das Verstandniss des
Mechanismus ist anfangs recht m�hsam und bedarf viel Ziel. Eine weitere
Problematik kommt aus der Tatsache, dass der Android Emulator, der mit dem
Android SDK mitgelieft wird, keine Wlan Unterst�tzung besitzt und somit die
Entwicklung mithilfe des Emulator nicht m�glich ist. Dies f�hrt dazu, dass f�r
die Entwicklung von Mehrspielerspielen mindestens 2 Android Ger�te notwendig
sind, um das Spiel vern�nftig testen zu k�nnen. Daher war es auch schwer
Belastungstest durchzuf�hren, da daf�r eine entsprechende Anzahl an Smartphones
ben�tigt wird. Alternativ g�be es noch die M�glichkeit eine Android x86
Version\cite{andx86} auf einer Virtual Machine laufen zu lassen, jedoch ist die
native Bibliothek von AllJoyn nur f�r die ARM-Architektur verf�gbar. Doch auch hier
gibt es eine M�glichkeit AllJoyn f�r die x86 Architektur zu bauen, da der
Quellcode frei verf�gbar ist, das jedoch einen gro�en Aufwand und die
Auseinandersetzung mit dem Android Quellcode bedeutet. Es gibt noch viel
Entwicklungspotential f�r die hier vorgestellte Bibliothek, sowie das verwendete
Rahmenwerk AllJoyn. Es fehlt z.B. die M�glichkeit ein Spiel �ber die Bluetooth
Verbindung zu spielen. AllJoyn bietet eine Bluetooth Unterst�tzung nur f�r
Smartphones mit \ib{root}\cite{root} Rechten. Weiterhin w�re eine automatische
Erstellung eines AccessPoints zu den sich die Spieler auch automatisch verbinden
ein interessantes Feature. Weiterhin w�re die Entwicklung der Bibliothek auch
f�r das iOS m�glich, sodass auch die Entwicklung von Peer-to-Peer Spielen f�r
das iPhone und iPad erleichtert wird. Als eine Opensource Alternative zu
Samsungs Chord SDK, hat diese Bibliothek das Potential, bei der Entwickung von
Peer-to-Peer Spielen auf dem Android, einen gro�en Beitrag zu leisten.
