\chapter{Implementierung}

Es gibt viele verschiedene Spielprinzipien und daher sollte soweit es geht mit
dem PTPRahmenwerk den sp�teren Entwicklern so viel Freiheit gelassen werden wie
m�glich.
Jedoch gibt es viele Abl�ufe, die sich immer wieder wiederholen. So hat man den
typschen Ablauf, dass man das Spiel startet und man sich in einer Lobby
befindet. Danach kann man entweder ein Spiel erstellen und sich zu einem bereits
erstellten Spiel verbinden. Diesen Ablauf wird durch das PTPRahmenwerk
�bernommen, sodass man sich hinterher darum nicht mehr k�mmern muss.
Der Entwickler h�tte dann nur noch die Aufgabe sich um die Activities zu
k�mmern, die das Spiel representieren und die Verbindung und Vermittlung w�rde
das PTPRahmenwerk �bernehmen. Um das Framework benutzen zu k�nnen m�sste der
Entwickler dann nur noch eine Jar-Datei als Bibliothek in sein Projekt einbinden und er
k�nnte loslegen.

\section{PTPHelper}
Nach Abw�gen mehrerer M�glichkeiten wie man den Service, welcher sich um die
Verbindungstechnischen Abl�ufe k�mmert, in die Applikation integrieren kann, ist
die Wahl schlie�lich auf die Benutzung eines Singletons gefallen. Zwar lie�e
sich eine Abstrakte Application benutzen, die der Entwickler in seinem Projekt
implementieren m�sste, aber dies w�rde dem Entwickler auch eine
Architekturentscheidung aufzwingen. Au�erdem ist eine zu enge Bindung an das
Android Rahmenwerk unvorteilhaft, da sich dessen API recht oft �ndert und die
Entwicklung in der Zukunft erschweren kann. Durch das Singleton kann der
Entwickler entscheiden, wann er den Service startet und wie er damit
interagieren will. So kann er z.B. den Einzelspielermodus vollkommen ohne den Helper realisieren und
erst bei Multiplayermodus den Service starten. Um den PTPHelper zu
initialisieren muss der Entwickler in seinem Code den Folgenden Befehl ausf�hren:

\begin{lstlisting}
PTPHelper.initHelper(MyInterface.class, context, proxyObject, signalHandler,
MyService.class, MyLobby.class);
\end{lstlisting}

Daraufhin kann er �ber den Getter an die initialisierte Instanz kommen.

\begin{lstlisting}
PTPHelper.getInstance();
\end{lstlisting}
Die P2PHelper Klasse ist eine generische Klasse und muss deshalb bei der
Instanziierung den entsprechenden Typ bekommen. Daher auch die etwas
umfangreiche Init-Methode. Es wird zum einen die Interface Klasse ben�tigt �ber
das die Kommunikation mit den anderen Peers realisiert wird. Bei der
Initialisierung wird der P2PService im Hintergrund gestartet welcher die Verbindung zum Bussystem aufbaut.
Das PTPRahmenwerk ist dazu ausgelegt nur im WLAN zu funkionieren, somit wird bei der
Initialisierung gepr�ft ob das Ger�t eine WiFi Verbindung momentan hat. 
Dies wird �ber den WiFiManager realisiert.

\begin{lstlisting}
WifiManager wifiManager =(WifiManager)context.getSystemService(Context.WIFI_SERVICE);
WifiInfo currentWifi = wifiManager.getConnectionInfo(); 
if((currentWifi==null || currentWifi.getSSID()== null || currentWifi.getSSID().isEmpty()){
	//Show Message there is no WiFi-Connection
}
\end{lstlisting}
Im Falle dessen, dass es keine WiFi-Verbindung gibt, wird die Initialisierung
abgebrochen und der Service wird nicht gestartet.
Android erm�glicht es zus�tzlich ein AccessPoint zu erstellen, wor�ber sich
andere Ger�te verbinden k�nnen. Die Information �ber diesen Zustand wird jedoch
nicht ohne weiteres �ber den WiFi-Mananger herausgegeben, sodass man an diese Information nur �ber
Java-Reflection gelangt:

\begin{lstlisting}

Method method = wifiManager.getClass().getMethod("isWifiApEnabled");
state = (Boolean) method.invoke(wifiManager);

\end{lstlisting}
Nach dem jedoch der Helper initialisiert ist und der Service gestartet, kann man
�ber den Helper den Service mitteilen sich z.B. zu einem Channel zu verbinden
oder einen Channel zu erstellen.
Dies geschickt �ber einfache Methodenaufrufe:

\begin{lstlisting}
PTPHelper.getInstance().setHostChannelName(name)
PTPHelper.getInstance().hostStartChannel()
...
PTPHelper.getInstance().setClientChannelName("channelName");
PTPHelper.getInstance().joinChannel();

\end{lstlisting}

\section{Graphenspiel}

 

\section{MauMau}

 
